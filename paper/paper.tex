\documentclass[9pt,twocolumn,twoside]{gsajnl_modified}
% Use the documentclass option 'lineno' to view line numbers

\usepackage[htt]{hyphenat}  % https://tex.stackexchange.com/a/543
\usepackage[export]{adjustbox}
\usepackage{xurl}
\usepackage{stfloats}

\renewcommand{\topfraction}{0.9}	% max fraction of floats at top
    \renewcommand{\bottomfraction}{0.8}	% max fraction of floats at bottom
    %   Parameters for TEXT pages (not float pages):
    \setcounter{topnumber}{2}
    \setcounter{bottomnumber}{2}
    \setcounter{totalnumber}{4}     % 2 may work better
    \setcounter{dbltopnumber}{2}    % for 2-column pages
    \renewcommand{\dbltopfraction}{0.9}	% fit big float above 2-col. text
    \renewcommand{\textfraction}{0.07}	% allow minimal text w. figs
    %   Parameters for FLOAT pages (not text pages):
    \renewcommand{\floatpagefraction}{0.7}	% require fuller float pages
	% N.B.: floatpagefraction MUST be less than topfraction !!
    \renewcommand{\dblfloatpagefraction}{0.7}	% require fuller float pages

\title{An antibody-escape calculator for mutations to the SARS-CoV-2 receptor-binding domain}

\author[*]{\Large Allison J. Greaney$^{1,2,3}$, Tyler N. Starr$^{1,4}$, Jesse D. Bloom$^{1,2,4}$}

\affil[1]{Basic Sciences and Computational Biology, Fred Hutchinson Cancer Center

} 
\affil[2]{Department of Genome Sciences, University of Washington

}
\affil[3]{Medical Scientist Training Program, University of Washington

}
\affil[4]{Howard Hughes Medical Institute

Seattle, WA, USA
}

\keywords{}

\runningtitle{} % For use in the footer 
\runningauthor{}

\begin{abstract}
A key goal of SARS-CoV-2 surveillance is to rapidly identify viral variants with mutations that reduce neutralization by polyclonal antibodies elicited by vaccination or infection.
Unfortunately, direct experimental characterization of new viral variants substantially lags their sequence-based identification. 
Here we help overcome this challenge by providing an ``escape calculator'' that estimates the antigenic effects of arbitrary combinations of mutations to the virus's spike receptor-binding domain (RBD).
Specifically, the calculator aggregates deep mutational scanning data to visualize how mutations impact polyclonal antibody recognition.
It also quantitatively scores how much each mutant is expected to escape RBD-targeted antibody neutralization.
These scores correlate with neutralization assays performed by multiple labs on a range of SARS-CoV-2 variants, and emphasize the ominous antigenic properties of the mutations in the recently described Omicron variant.
An interactive version of the calculator is at \url{https://jbloomlab.github.io/SARS2_RBD_Ab_escape_maps/escape-calc/}, and we provide a Python module for batch processing.
\end{abstract}

\begin{document}

\maketitle
\thispagestyle{firststyle}
%\marginmark
\firstpagefootnote

\correspondingauthoraffiliation{}{*Corresponding author: \href{mailto:jbloom@fredhutch.org}{jbloom@fredhutch.org}}
\vspace{-33pt}% Only used for adjusting extra space in the left column of the first page

\lettrine[lines=2]{\color{color2}H}{}uman coronaviruses undergo rapid antigenic evolution that erodes antibody-based neutralization~\citep{eguia2021human,kistler2021evidence}.
This antigenic evolution is already apparent for SARS-CoV-2, as a variety of new viral variants with reduced antibody neutralization have already emerged only $\sim$2 years since the virus first started to spread in humans~\citep{?}.
A tremendous amount of experimental effort has been expended to characterize these SARS-CoV-2 variants in neutralization assays~\citep{?}.
Unfortunately, the rate at which new variants arise and spread can outstrip the speed with which these laborious experiments can be performed.

A partial solution is to use deep mutational scanning experiments to \emph{prospectively} characterize how viral mutations impact antibody binding or neutralization.
These experiments can systematically map the antigenic impacts of all possible amino-acid mutations in key regions of spike on specific antibodies or serum samples~\citep{?}.
However, SARS-CoV-2 variants of concern typically have many mutations in spike, and it is not feasible to experimentally characterize all combinations of mutations even via high-throughput approaches such as deep mutational scanning.

Here we provide a partial solution to this challenge by aggregating deep mutational scanning data across many antibodies to assess the impacts of combinations of mutations in the spike receptor-binding domain (RBD), which is the primary target of neutralizing antibodies to SARS-CoV-2~\citep{?}.
The resulting ``escape calculator'' enables both qualitative visualization and quantitative scoring of the antigenic impacts of arbitrary combinations of RBD mutations. 

\section{Results}

\subsection{}

\section{Discussion}


{\small

\section{Methods}
\subsection{Code and data availability}

\section{Acknowledgments}
{\color{red} GISAID}
JDB is an Investigator of the Howard Hughes Medical Institute.

\section{Competing interests}
JDB consults for Moderna and Flagship Labs 77, and is an inventor on a Fred Hutch licensed patents related to deep mutational scanning of viral proteins.

}

\bibliography{references}

\end{document}
